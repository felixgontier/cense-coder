\documentclass[10pt]{article}

\title{Reply to reviewers concerning submission sensors-238686: "An efficient audio coding scheme  for quantitative and qualitative large scale acoustic monitoring using the sensor grid approach"}

\begin{document}

\maketitle

As a preamble, we would like to thank the editor and the reviewer for their comments and suggestions. Following these comments, we made several changes to the article, which are summarized here. The next sections list our answers to each of the reviewer’s comments, with references to the revised manuscript (page, column, and paragraph) where appropriate.


\section{Answers to Reviewer I}

\begin{enumerate}

\item \emph{It is recommended to discuss about noise annoying and loudness monitoring and evaluation with sensor network.}

$\rightarrow$ 

\item \emph{Is it possible to achieve a two channel coding and transmission with this proposed scheme?}

$\rightarrow$ 

\item \emph{Typos:\\
- Small English typos.\\
- Use of "we" too often. It is recommended using third person writting.\\
- Equation numeration}

$\rightarrow$ 

\end{enumerate}

\section{Answers to Reviewer II}

\begin{enumerate}

\item \emph{The Introduction is a bit shaky and needs to be reworked in my opinion. This section should: a) provide a literature review of the state of the art on acoustic sensors’ networks; b) identify a methodological gap in current practice; c) explain why it is important to fill it; and d) give a glimpse about how the proposed research aimed to do so. The current structure is confusing to me, with Section 2 coming a bit out of the blue (I feel it should be reported earlier in the Introduction).}

$\rightarrow$

\item \emph{Please, also note that not all readers are likely to be familiar with ASC and AED protocols, so it would be nice if the authors could provide some more details about the general steps; i.e., feature extraction (parametrisation) stage, classification stage, etc.}

$\rightarrow$

\item \emph{The Validation protocol (Section 4) is not very convincing. In particular, the subjective validation of Section 4.3 does not look rigorous at all.}

$\rightarrow$

\item \emph{In general, a strange aspect for me is that, this paper being submitted to the journal Sensors, there is no actual description of the sensors’ network architecture and/or hardware components proposed for the current study.}

$\rightarrow$

\item \emph{The overall structure of the paper should be re-thought in my opinion, as the reader struggles to orientate himself between the different sections of the manuscript. When I read the conclusions, it is hard to track back the discourse in the methodological and results sections.}

$\rightarrow$

\end{enumerate}


\section{Answers to Reviewer III}

\begin{enumerate}

\item \emph{It would be nice if the authors could add the following information to their work: Motivation for using the third octave band spectral representation.}

$\rightarrow$

\item \emph{Clearly state the novel points of this work in the introduction.}

$\rightarrow$

\item \emph{Why did you select these classification schemes? There many approaches in the literature exploiting that; for example "A novel holistic modeling approach for generalized sound recognition". In addition to that, the authors should justify why the did not employ deep learning (for both feature extraction and pattern recognition).}

$\rightarrow$

\item \emph{Another point regarding detection would be to relate this work with other operating in real unrestricted environments, such as "Acoustic detection of human activities in natural environments" and how the proposed method could boost the existing ones.}

$\rightarrow$

\item \emph{Since the authors use the UrbanSound8k dataset, it is not clear how the sensor grid approach is applied. Is there a grid of sensors included?}

$\rightarrow$

\item \emph{The authors should compare their system with other existing in the related literature to understand its merits and limitations. This should be done in many levels, i.e. coding, recognition, detection, etc.}

$\rightarrow$

\item \emph{Regarding intelligibility, it would be nice if the authors gave more information on the used metric. Moreover they should use standard metrics used for that which are similar to those used in the speech enhancement field. See paper "Objective comparison of speech enhancement algorithms under real world conditions" for more details.}

$\rightarrow$

\end{enumerate}


\end{document}
